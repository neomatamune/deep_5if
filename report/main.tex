\documentclass[french]{article}
\usepackage[utf8]{inputenc}
\usepackage[T1]{fontenc}
\usepackage{babel}
\usepackage{url}

\usepackage{pdfpages}
\usepackage{framed}
\usepackage{hyperref}
\usepackage{graphicx}
\usepackage{ulem}
\usepackage{adjustbox}
\usepackage{blindtext}
\usepackage{tocbibind}
\usepackage[toc,page]{appendix}
\PassOptionsToPackage{hyphens}{url}

\usepackage{minted}

\title{Face recognition with convoluted neural networks
\\-
\\OT1-PJ12}
\author{FOSSART Alexis, RENAULT Benoit}
\date{January 31st 2018}

%% Margins
\usepackage[
    top=3cm,
    bottom=3cm,
    left=3cm,
    right=3cm
]{geometry}

\begin{document}

\maketitle

\medskip

\tableofcontents

\newpage

\section{Introduction}



\section{Running the project}

\subsection{Prerequisites}

Please run it preferably on a GNU/Linux distribution. You should also have installed the latest version available to you of Git and Docker (google them to find their installation instructions for your OS).

\subsection{Quickstart}

\paragraph{} Clone this repository:

\begin{minted}{bash}
git clone
\end{minted}

\paragraph{} Run the build.sh, then start\_jupyter, both with sudo rights if you haven't set your user so that you can use docker commands without being root.

\begin{minted}{bash}
sudo ./build.sh
sudo ./start_jupyter.sh
\end{minted}

\paragraph{} The first command log should end with a message like the following:

\end{minted}
Successfully built 554f78aa2d91
Successfully tagged deep:latest
\end{minted}

\paragraph{} The second one should return something like:

\end{minted}
52aac7bf9976a85dd9b98258dc350bbe790acb94152c33884f6b3885a21bf31a
\end{minted}

\paragraph{} You can check whether the container started properly by writing:

\begin{minted}{bash}
sudo docker ps -a
\end{minted}

\paragraph{} If everything is fine you should get a line like the following (what matters here is that the "STATUS" field):

\end{minted}
CONTAINER ID        IMAGE               COMMAND                  CREATED             STATUS                    PORTS                    NAMES
52aac7bf9976        deep                "/bin/sh -c 'jupyter…"   13 minutes ago      Up 13 minutes             0.0.0.0:8887->8888/tcp   deep_jupyter

\end{minted}

\paragraph{} Once this is done, you should have a "data" folder in your cloned repository. Copy the test_images and train_images data folders into it. Then open a bash in the container :

\begin{minted}{bash}
sudo docker exec -it deep_jupyter /bin/bash
\end{minted}

\paragraph{} And run the following commands inside to build the image database :

\begin{minted}{bash}
cd data

/opt/caffe/build/tools/convert_imageset --shuffle --gray train_images/ posneg.txt train_lmdb

/opt/caffe/build/tools/convert_imageset --shuffle --gray test_images/ testposneg.txt test_lmdb

/opt/caffe/build/tools/compute_image_mean train_lmdb/ train_mean.binaryproto

/opt/caffe/build/tools/compute_image_mean test_lmdb/ test_mean.binaryproto
\end{minted}

\paragraph{} You can then close your terminal and open your favorite web browser to http://localhost:8887/ . This will open the Jupyter Notebook instance that is inside the container.

\paragraph{} You will get to a login form. In the "Password or token" field, enter "demo". You will then be redirected to the usual Jupyter Notebook interface. Open the "notebooks" folder, and then the "01_CNN_Training.ipynb" if you want to train the neural network, or "02_CNN_Usage.ipynb" if you want to use a pre-trained one or your own.

\section{Description of the work}



\subsection{Setting up Caffe to be used through PyCaffe and Docker}

\subsection{Training the network}

\subsubsection{Code}

\begin{minted}{python}
import numpy as np
import matplotlib.pyplot as plt
from PIL import Image
import caffe

caffe.set_mode_cpu()
solver = caffe.SGDSolver('solver.prototxt')
solver.solve()
\end{minted}

\subsubsection{Adjusting the Docker image for better performances during training}

\subsection{Evaluating the detection performance of our NN}

\subsubsection{Code}

\paragraph{Basic setup of neural network}

\begin{minted}{python}
import caffe

caffe.set_mode_cpu()

model_train = 'conv.prototxt'
model_test = 'deploy.prototxt'
weights = 'deep_iter_100000.caffemodel'

net_training = caffe.Net(model_train, weights, caffe.TEST)
net_testing = caffe.Net(model_test, weights, caffe.TEST)
\end{minted}

\paragraph{Compute the accuracy of the previously trained neural network on the training dataset (see 01\_CNN\_Training notebook) by using the accuracy layer of the network (defined in conv.prototxt)}

\begin{minted}{python}
NUMBER_OF_IMAGES_ON_TRAINING_SET_WITHOUT_FACE = 26950
NUMBER_OF_IMAGES_ON_TRAINING_SET_WITH_A_FACE = 64770

TOTAL_NUMBER_OF_IMAGES_ON_TRAINING_SET = (NUMBER_OF_IMAGES_ON_TRAINING_SET_WITHOUT_FACE +
                                          NUMBER_OF_IMAGES_ON_TRAINING_SET_WITH_A_FACE)

total_accuracy = 0
batch_size = net_training.blobs['data'].num
test_iters = TOTAL_NUMBER_OF_IMAGES_ON_TRAINING_SET / batch_size

for i in range(test_iters):
    net_training.forward()
    batch_accuracy = net_training.blobs['accuracy'].data
    total_accuracy += batch_accuracy
accuracy = total_accuracy / test_iters

print "Accuracy of trained network on train data: {}".format(accuracy)
print "Number of well classified images on train data: {}".format(
    int(accuracy * TOTAL_NUMBER_OF_IMAGES_ON_TRAINING_SET))
\end{minted}

\paragraph{Evaluate the accuracy of the previously trained neural network on the test dataset}

\begin{minted}{python}
NUMBER_OF_IMAGES_ON_GOOGLEFACE_TEST = 632
NUMBER_OF_IMAGES_ON_GOOGLE_IMAGES = 6831
NUMBER_OF_IMAGES_ON_YALEFACES_TEST = 165

TOTAL_NUMBER_OF_IMAGES_ON_TEST_SET = (NUMBER_OF_IMAGES_ON_YALEFACES_TEST +
                                      NUMBER_OF_IMAGES_ON_GOOGLE_IMAGES +
                                      NUMBER_OF_IMAGES_ON_GOOGLEFACE_TEST)

img_classified_as_faces = 0
img_classified = 0

total_accuracy = 0

batch_size = net_testing.blobs['data'].num
test_iters = TOTAL_NUMBER_OF_IMAGES_ON_TEST_SET / batch_size

for i in range(test_iters):
    net_testing.forward()
    batch_accuracy = net_testing.blobs['accuracy'].data
    total_accuracy += batch_accuracy
    for i in range(batch_size):
        img_classified += 1
        if net_testing.blobs['prob'].data[i].argmax():
            img_classified_as_faces += 1
            

accuracy = total_accuracy / test_iters

print "Number of faces found on test : {} / {} images".format(img_classified_as_faces,
                                                              img_classified)
print "Accuracy of trained network on test data: {}".format(accuracy)
print "Number of well classified images on test data: {}".format(
    int(accuracy * TOTAL_NUMBER_OF_IMAGES_ON_TEST_SET))
\end{minted}

\subsection{Implementing a naive face detector}

\paragraph{Basic setup of neural network}

\begin{minted}{python}
import numpy as np
import matplotlib.pyplot as plt
from PIL import Image
import cv2
import caffe
import os

caffe.set_mode_cpu()
model = 'deploy.prototxt'
weights = 'deep_iter_100000.caffemodel'
net = caffe.Net(model, weights, caffe.TEST)

try:
    os.mkdir("../data/results/")
except OSError: pass
try:
    os.mkdir("../data/results/1/")
except OSError: pass
\end{minted}

\paragraph{Implementation of face detector} Downscaling done by factor of 2 (using an integrated function of opencv which seems to be the only way to do pyramid scaling without destroying the quality of the images). Naive algorithm implementation, without heuristics : we simply use a shifting window of size 36*36px and offset 4px.

\begin{minted}{python}
for id_img in range(1, 8):
    # Used on 7 images named 1.jpg to 7.jpg
    image_path = '../data/' + str(id_img) + '.jpg'
    
    # Open image and convert to gray scale
    im =  cv2.imread(image_path)
    im = cv2.cvtColor(im, cv2.COLOR_BGR2GRAY)
    
    # Save base image for future comparison
    imbase = im
    
    scale = 1
    
    base_save_path = "../data/results/"
    img_base_name = str(id_img) + "_"
    
    # Used to batch save detected faces, and avoid writing to disk at every loop
    keep = dict()
    
    # Tuple with (width, height)
    shifting_window_size = 36, 36
    offset = 4
    
    min_probability_for_a_match = 0.99
    
    while len(im) >= shifting_window_size[0] * 2 and len(im[0]) > shifting_window_size[1] * 2:
        # Downscale of the image using opencv pyramid scaling
        im = cv2.pyrDown(im)
        scale*=2
        
        img_scale_name = img_base_name + str(scale) + "_"
        
        showarray(im)
        
        # Face detector
        for x in range(0, len(im) - shifting_window_size[0], offset):
            for y in range(0, len(im[0]) - shifting_window_size[1], offset):
                img_name = img_scale_name + str(x) + "_" + str(y) + ".png"
                
                # Create the shift window image on the original scaled image 
                imtmp = np.array(im [x:x + shifting_window_size[0], y:y + shifting_window_size[1]])
                
                # Transform the data to be compatible as an entry to the neural network
                im_input = imtmp[np.newaxis, np.newaxis, :, :] / 256.0
                
                # Input the data in the NN
                net.blobs['data'].reshape(*im_input.shape)
                net.blobs['data'].data[...] = imtmp
                
                # Run the NN
                output = net.forward()
                
                # If the probability that the image that is in the
                #shift window is a face is higher than 99%
                if output['prob'][0][1] > min_probability_for_a_match:
                    # We add a bounding box in the original image
                    white = 255
                    for i in range (x * scale, (x + shifting_window_size[0]) * scale):
                        imbase[i][y * scale] = white
                        imbase[i][(y + shifting_window_size[1]) * scale] = white
                    for j in range(y * scale, (y + shifting_window_size[1]) * scale):
                        imbase[x * scale][j] = white
                        imbase[(x + shifting_window_size[0]) * scale][j] = white

                    # Save this in memory
                    save_dir = "1/"
                    save_path = base_save_path + save_dir + img_name
                    keep[save_path] = imtmp
    
    # When everything is done we show the original image
    #with the bounding boxes and save it as a file
    showarray(imbase)
    save_results = "../data/results/" + str(id_img) + ".jpg"
    cv2.imwrite(save_results, imbase)
    
    # We also save the face subimages we found to improve the NN in the future
    for path,img in keep.iteritems():
        cv2.imwrite(path, img)
\end{minted}

\section{Conclusion}

\end{document}
